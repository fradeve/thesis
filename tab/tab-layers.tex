\begin{table}
    \centering
    \begin{tabular}{cccc}
        \toprule
        Layer type & Description          & Features & Scale \\
        \otoprule
        raster     & magnetic prospection & ---      & TODO  \\
        raster     & aerial photography   & ---      & TODO  \\
        vector     & ditches / compounds / areas & 1300   & ---   \\
        \bottomrule
    \end{tabular}
    \caption[List of layers in Tavoliere neolithic GIS project]{Details of the layers in the Tavoliere GIS project.
    \label{tab:layers}
    }
\end{table}
% tif con griglia magnetogramma
% magnetogramma tif
% foto ortorettificata
% layer vect punti GPS estremi magnetogramma

% viene prima inserito magnetogramma con griglia in unico file raster, poi derivato magnetogramma senza griglia
% georeferenziazione viene operata su magnetogramma con griglia

% WMS SIT Puglia
% WMS Geoportale Nazionale
% cartografia IGM 25000
% carta geologica
