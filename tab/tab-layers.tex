\begin{table}
    \centering
    \begin{tabular}{cccc}
        \toprule
        Layer type & Description          & Features & Scale \\
        \otoprule
        raster     & magnetic prospection grid & --- & TODO \\
        raster     & magnetic prospection & ---      & TODO \\
        raster     & aerial photography   & ---      & TODO \\
        raster     & WMS SIT Puglia       & ---      & TODO \\
        raster     & WMS Geoportale Nazionale & ---  & TODO \\
        raster     & IGM cartography      & ---      & 1:25000 \\
        raster     & geologic cartography & ---      & TODO \\
        vector     & magnetic prospection vertexes & --- \\
        vector     & ditches / compounds / areas & 1300   & --- \\
        \bottomrule
    \end{tabular}
    \caption[List of layers in Tavoliere neolithic GIS project]{Details of the layers in the Tavoliere GIS project.
    \label{tab:layers}
    }
\end{table}

% viene prima inserito magnetogramma con griglia in unico file raster, poi derivato magnetogramma senza griglia
% georeferenziazione viene operata su magnetogramma con griglia
