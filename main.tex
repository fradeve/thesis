%& -shell-escape
\documentclass[12pt,%                     % corpo del font principale
               a4paper,twoside,		      % carta A4, fronte
               titlepage,%                % frontespzio
               headinclude,,footinclude,% % testatina e pié di pagina
               numbers=noenddot,%         % niente punto dopo il numero delle sezioni
               cleardoublepage=empty,%    % pagine vuote senza testatina e pié di pagina
               abstracton,%
               appendixprefix=true,%
               %draft                     % useful for debug
               ]{scrreprt}                % classe report di KOMA-Script;

%%% font / input %%%%%%%%%%%%

\usepackage[T1]{fontenc}                  % la codifica dei font
\usepackage[utf8]{inputenc}               % imposta la codifica di input;

%%% language %%%%%%%%%%%%%%%%

\usepackage[italian,english]{babel}       % per scrivere in italiano e in inglese;

%%% document style %%%%%%%%%%

\usepackage[eulerchapternumbers,%         % numeri dei capitoli in Euler
            subfig,%                      % compatibilità con subfig
            eulermath,%                   % AMS Euler come font per la matematica
            %pdfspacing%                  % migliora il riempimento di riga con PDFLaTeX
            ]{classicthesis}              % lo stile ClassicThesis
\usepackage{arsclassica}                  % modifica alcuni aspetti di ClassicThesis
\usepackage{setspace}                     % setting space among lines
\onehalfspacing
\KOMAoptions{cleardoublepage=plain}       % the cleardoublepage is empty

%%% bibliography %%%%%%%%%%%%

\usepackage[autostyle]{csquotes}          % useful with biblatex
\usepackage[backend=biber,                % bibliography
            bibstyle=alphabetic,          % .
			citestyle=alphabetic,         % .
			hyperref,                     % .
			useprefix,                    % .
			backref]{biblatex}            % .
\addbibresource{refs/bibliografia.bib}    % loads bib file

%%% math and science %%%%%%%%%

\usepackage{amsmath,amssymb,amsthm}       % indispensabili per la matematica
\usepackage{xfrac}                        % slanted bar for fractions
\usepackage{siunitx}	                  % SI units
\usepackage{eurosym}                      % euro symbol

%%% layout %%%%%%%%%%%%%%%%%%

\usepackage[{a4paper,                     % margins
             left=3cm,                    % .
			 right=2.5cm,                 % .
             top=3cm,                     % .
			 bottom=3.5cm,                % .
             heightrounded,               % .
			 bindingoffset=1.5cm}]{geometry}
\usepackage{varioref}                     % riferimenti completi della pagina
\usepackage[plain]{fancyref}
%\usepackage{indentfirst}                 % rientra il primo paragrafo di ogni sezione
\usepackage{minitoc}
\usepackage{wrapfig}

%%% tables %%%%%%%%%%%%%%%%%%%

\usepackage{booktabs}			          % migliora la qualità delle tabelle
\usepackage{tabularx}			          % colonne a spaziatura fissa delle tabelle
\newcommand{\otoprule}                    % better top rule horizontal line
    {\midrule[\heavyrulewidth]}           % .

%%% images and graphics %%%%%%

\graphicspath{{img/}}                     % defines images path
\usepackage{graphicx}                     % basic graphics support
\usepackage{pdfpages}                     % include pdf
\usepackage{subfig}                       % sottofigure, sottotabelle
\usepackage{sidecap}                      % enables captions on the side
\usepackage[tableposition=top]{caption}   % define captions style
\usepackage{float}
\usepackage{parskip}

%% tikz
\usepackage{dot2texi}
\usepackage{tikz}
\usetikzlibrary{shapes,arrows}
\usepackage{rotating}
\usepackage[dvipsnames]{xcolor}

\newcommand\AlCentroPagina[1]{            % generates big b/w logo
    \AddToShipoutPicture*{\AtPageCenter{  % .
        \makebox(0,0){\includegraphics    % .
            [width=0.9\paperwidth]{#1}}   % .
        }                                 % .
    }                                     % .
}                                         % .

%%% text utils %%%%%%%%%%%%%%%

\usepackage{hyphenat}
\hypersetup{colorlinks=false}		      % commentare se il file non è x la stampa
\usepackage{acronym}			          % pacchetto per gli acronimi

% generates summary (before starting chapter's first section)
\newenvironment{chaptersum}
    {\sffamily\textbf{Sommario}\\\rightskip1in}
    {}


\begin{document}
    \includepdf[pages={1}]{pages/tesi-front}
    \cleardoublepage
    \includepdf[pages={1}]{pages/tesi-post-front}
    \cleardoublepage
    %******************************************
    % Materiale iniziale
    %******************************************
    \pagenumbering{Roman}
    \thispagestyle{empty}
\begin{flushright}
    \null\vspace{\stretch {1}}
        To my family,\\for having given me\\the tranquility for thinking
    \vspace{\stretch{2}}\null
\end{flushright}
\vspace{-1.3\textwidth}
\begin{flushright}
    \null\vspace{\stretch {1}}
        To Tina\\far together
    \vspace{\stretch{2}}\null
\end{flushright}
\cleardoublepage
    \pagestyle{scrheadings} 
    \pdfbookmark{\contentsname}{tableofcontents}
\setcounter{tocdepth}{2}
\dominitoc\tableofcontents

\pdfbookmark{\listfigurename}{lof}
\listoffigures

\pdfbookmark{\listtablename}{lot}
\listoftables
\cleardoublepage
    %\chapter*{Forewords}

    some nice fore text
\cleardoublepage
    %******************************************
    % Materiale principale
    %******************************************
    \pagenumbering{arabic}
    \chapter{Introduction}

    \begin{chaptersum}
        A 35 anni dal primo utilizzo dei GIS in archeologia, questi costituiscono oggi uno strumento indispensabile per la gestione della conoscenza archeologica esistente e per la derivazione di nuovi dati utilizzando calcoli statistici e metodologie di analisi spaziale. A questa tendenza si affianca il sempre crescente ruolo che l'analisi geofisica ha nella conoscenza del territorio e delle strutture sepolte (\textsection~\ref{sec:arch-geoph}). A fronte degli evidenti vantaggi che entrambe le discipline portano alla ricerca archeologica e scientifica (riduzione dei costi e tempi di analisi e gestione della conoscenza), emergono alcuni problemi relativi alla gestione dei dati geografici ricavati (\textsection~\ref{sec:gis-data-management}). La crescente quantità e qualità dei dati ottenuti con i metodi di ricerca moderni rende indispensabile la definizione di nuovi percorsi di gestione, basati su tecnologie che permettano di ridurre i tempi di derivazione, analisi e interrogazione di grandi quantità di dati. Il lavoro corrente presenta un software che automatizza il calcolo di alcune strutture derivate da dati geografici, e permette di ridurre drasticamente il tempo necessario ad ottenere nuovi dati spaziali e statistiche da grandi set di dati, sostituendo parte del lavoro manuale dell'operatore con algoritmi che utilizzano principi geometria analitica, avvicinando l'approccio adottato fino ad oggi ad un nuovo percorso più vicino all'archeologia del paesaggio (\textsection~\ref{sec:landscape}). Il caso di studio analizzato è quello di alcuni insediamenti neolitici del Tavoliere.
    \end{chaptersum}

    The history of non-invasive geophysical research and methods has evolved rapidly over the years, widening the horizons of research in all the fields in which it has been applied. In this context, archaeology is one of the disciplines which has taken advantage of this and the number of published experiences on the topic is growing every year.\\
    In the last 20 years, both geophysics and archaeology have steadily expanded their use of Geographic Information Systems (GIS) to organize, clean, analyze and produce results from raw data, and 35 years have past since the first application of GIS in archaeology \cite[pp.~2--3]{kvamme1995}. From this point of view, GIS placed itself in the intersection of both fields and has developed itself as a separate discipline over the years, moving from desktop software to the internet with \emph{webGIS}. The ever growing amount of case studies focused on GIS, geophysics and archaeology is making clear that in the future these disciplines will evolve together towards an interdisciplinary approach, and probably each of these will take advantage from the advancements in the others.\\
    This works presents a case study of a statistical approach on geophysical data recorded in the archaeological field, using some well known methods and introducing some innovations with respect to data analysis, webGIS and interfaces to present results. 

    \section{Archaeology and geophysics}
    \label{sec:arch-geoph}
        The relationship between geophysics and \emph{landscape archaeology} in particular is well documented and presents some practical advantages over any other approach, mainly:
        \begin{description}
            \item[cost reduction] geophysical surveys can drive the excavation and avoid losing time in unfruitful diggings, helping to define the best approach to manage the site
            \item[flexible approaches] geophysical methods are suitable for large scale analysis as well as small scale surveys
            \item[integration] since part of the archaeological data consists of spatial data, geophysical information --- most of the times --- could be easily integrated with current data, defining another ``layer of knowledge''.
        \end{description}

    \section{Archaeology and Geographic Information Systems}
        The natural tendency of modern archaeology to move from small scale excavations to a larger view and comprehension of the relationship between ancient communities and past natural environment --- the so called \emph{landscape archaeology} --- has brought new methods in archaeology, and most of these relies on Geographic Information Systems. In this field, this relationship has gained strength more than with any other archaeological approach, as most of the landscape archaeology methods beyond excavation are \emph{spatial methods} \cite[preface]{space-archaeology}. The commonest uses of GIS in this field are: data classification, data plotting, environment modeling; the first two have been used in this work.

        By definition \cite{american-dict}, the scientific method consists of
        \begin{quote}
            the observation, identification, description, experimental investigation, and theoretical explanation of phenomena. Such activities restricted to a class of natural phenomena.
        \end{quote}
        Landscape archaeology deals with spatially described phenomena, and in modern approaches, GIS are the best tool to manage, identify and do experimental investigation on phenomena distributed on everything more complex than a Cartesian surface (like a geoid).

    \section{Geophysical analysis workflow}
        Despite of some peculiarities depending on the approach, the workflow of a geophysical analysis in the archaeological context has a fixed structure consisting of a sequence of data collecting techniques, cleaning of the collected data and GIS methods to clean, calculate and plot results. These steps, with particular attention to topics related to the current work, are briefly described below.

        \subsection{Aerial photography}
            The term \emph{aerial} refers to any image of the ground taken from an elevated position \cite[p.~452]{terrestrial-remote}

            \begin{wrapfigure}{l}{0.4\textwidth}
                \centering
                \vspace{-0.02\textheight}
                \begin{tikzpicture}[
    path image/.style={
        path picture={
            \node at (path picture bounding box.center) {
            \includegraphics[height=5cm]{#1}
            };
        }
    }
]
    \draw [path image=salerno-a,draw=black!70,thick] (0,0) circle (2.5cm) coordinate(aer);
    \draw [path image=salerno-c,draw=black!70,thick] (0,-6) circle (2.5cm) coordinate(magn);
    \draw [path image=salerno-d,draw=black!70,thick] (0,-12) circle (2.5cm) coordinate(vett);
\end{tikzpicture}

                \caption[The commontest 3 steps of the geophysical analysis.]{The commonest 3 steps of the geophysical analysis: aerial photography, magnetometry, manual vector data derivation on GIS.}
                \label{fig:magn-circles}
                \vspace{-0.14\textheight}
            \end{wrapfigure}

            In geophysics it is used to describe the images taken from aircrafts, helicopters or balloon with the specific purpose to define the characteristics of the ground before starting a deeper investigation with more precise methods.
            It is useful to have a general idea of where heritage is on the ground and its relations with the natural and anthropic environment.

            With reference to the nature of the work that will be presented in the following chapters, the aerial (oblique) survey gives best results in a plain farmed ground, providing effective documentation on cropmarks, classical and medieval structures (as well as their place in rural or urban landscape) \cite[pp.~11--12]{arch-site-detection}. Before satellite imaging, it has been one of the most valuable survey techniques and its use rapidly matured during the two World Wars.
            
        \subsection{Magnetometry}
            Magnetic techniques are proved as the most effective ones, when the necessity rises to define the position and size of buried structures. In its basic principles, it measures the anomalies on the measured Earth's magnetic field on the ground, caused by the presence of buried archaeological remains. Different configurations of the magnetometer's sensors are available (e.g.\ \emph{duo--quadro} sensors) and usually it is operated moving sensors on parallel equispaced lines (\num{0.1} to \SI{0.5}{\meter}), enabling the survey of large areas in a very small amount of time. The obtained data must be processed by specialized software (cleaning them by defects like \emph{spikes}, \emph{stripes} and \emph{zig-zag}) to obtain meaningful representations of the measures \cite[p.~462]{terrestrial-remote}; these representations give a detailed visual insight and 3D appearance of the underlying structures.

        \subsection{GIS and vector data creation}
            Analyzing data from different sources and with different contents like photos and magnetic prospecting would be impossible without the help of a spatial analysis software, which is in fact merging them in a single entity (the GIS project) using the only common attribute: the geographical coordinates. The data (after being georeferred and projected with a suitable geographical projection) are stored as \emph{raster} (actually, images with a latitude--longitude tuple associated with every pixel) in a plain file or database.\\

            \subsubsection{Deriving vectors from observation}
                As noted elsewhere, ``the human eye (and the brain behind) may be the best sensor for archaeological pattern recognition'' \cite[p.~164]{becker}. Having as base layers the information-rich rasters, the operator can draw GIS geometries (\emph{points}, \emph{lines} or \emph{polygons}) to represent the features visible in the raster layers. Notably, this is an interpretation itself: the operator is creating new knowledge starting from available data from different sources, assigning attributes to each feature to identify it (\emph{id}) or to classify it (\emph{road}, \emph{building}, \emph{ditch}). Moreover, geographical attributes are automatically stored in the newly created geometries, and all the GIS software are able to export them in common exchange formats (\emph{ESRI shapefile}, \emph{geoJSON}, etc.).

            \subsubsection{Deriving vectors from analysis}
                Most GIS suites integrate sets of functions calling algorithms to derive information from available data. E.g., a perimeter can be derived from a subset of all the features, or a buffer geometry can be created from a river basin's perimeter. All these data are geometries and could be saved in a separate layer, enriching the data collection (literature offers a great variety of these operations in a geophysical context \cite[p.~325]{remote-ciminale})
                . Depending on the GIS software used, functions may be implementations of well known --- published --- algorithms or written by the software's producer and distributed as \emph{proprietary} algorhitms; the latter case is the less advisable for the user, since he will know the results of the calculation without knowing the how the calculation itself has been operated \cite[p.~69]{fronza-informatica}.

        \subsection{GIS data management}
        \label{sec:gis-data-management}
            The workflow defined in this chapter has some points of weakness; some of them are related to the operator when deriving vectors from observation, others are limitations of the software suite itself:
            \begin{description}
                \item[errors related to magnification] when the GIS user derives vector data (usually polygons) from magnetic prospections, the zoom level set while editing the layer's features is fundamental for an accurate drawing; if the base layers are set at a zoom level lower than the adequate one, a rising of the difference between derived geometries and original raster layers will take place, heavily decreasing the accuracy and precision of the new data.
                \item[database management] collecting all the abovementioned data types for hundreds of sites will make the database grow heavily, raising the necessity to have an efficient data storage architecture. In the geospatial field, this usually means to stop using ESRI Shapefiles in favour of spatially-enabled database (Spatialite, PostGIS, etc.). The approach to these technologies is not simple most of the times, and lowering the barriers to access geospatial functions in database becomes a priority.
                %\item[results exportation] 
                \item[desktop software] most of the analysis process is usually done using functions integrated in a desktop GIS software, making very easy to move data across physical storage devices, but decreasing consistently the data safeness (since data needs to be backed up regularly and the machine itself can be damaged by other people in the laboratory). Moreover, giving external collaborators access to desktop data is very difficult, as well as to manage --- or log --- the access to data from a single person. These disadvantages are well documented in literature (see \cite[p.~19]{fronza-informatica}).
            \end{description}

    \section{Moving towards landscape archaeology}
    \label{sec:landscape}
        In the last years, the combination of stratigraphic archaeology, geophysical methods and GIS systems has brought tremendous advancements in the quantity and quality of the produced data. Time and cost effective remote sensing techniques, together with the mathematical and statistical methods introduced in the GIS suites have sped up the post processing and opened new ways to the comprehension of the data.

        GIS systems enable the operator to access and bulk analyze huge amounts of information, extracting meanings from statistical data not even conceivable without appropriate technologies. Nonetheless, these operations took place in the past (a bright example of how calculation where done 30 years ago in this field is shown in \cite{jones-tavoliere}) but implied long processing times, with consequences on the cost of the whole process.

        All these factors contributed to a radical redesign of the workflow, which now puts magnetometry beside aerial photography and automated statistics beyond manual data gathering and derivation. These new approaches to data gathering cope well and seem the natural extension of the \emph{three ``s''} of the geophysical prospections: speed, sensitivity and spatial resolution \cite[p.~130]{becker}, and lead us to the concept of \emph{landscape archaeology} as a comprehensive approach to these data.

        It becomes evident that this workflow generates heavy datasets, and some approaches still used today appear more inadequate to the extension and details of our current --- and future --- knowledge as the data mass grows. Trying to aid the human eye-brain system with automated statistics and data analysis seems the best way to increase and derive new knowledge from the available assets, and this work modestly points to give a contribution in this direction.

        \begin{figure}[H]
            \centering
            \resizebox{1\textwidth}{!}{%
                \begin{tikzpicture}
                    \input{tab/dot-flow-general}
                \end{tikzpicture}
            }
            \caption[Statistics from settlements: the Jones' approach, the current one and the proposed one]{Three different approaches to obtain statistics from neolithic settlements. In chronological order: the approach used by G.\ D.\ B.\ Jones (dashed line), the current approach (continuous black line) and the path proposed in this work (red line).}
            \label{fig:scheme-general}
        \end{figure}

        In \fref{fig:scheme-general} the three different solutions adopted during time to the above mentioned problems are shown. The current approach is considerably different from Jones' one \cite{jones-tavoliere}: while today vector layers can be derived using different data sources, 30 years ago the only valuable source of information about settlements were aerial photos (the work of G.\ D.\ B.\ Jones will be described in detail in \fref{sec:jones}). Moreover, statistics were derived directly from paper drawings (traced using aerial photos) and were stored as tables. Today, we are able to manage statistics using GIS systems.
        
        The proposed model, marked in red, will completely automate the derivation of geometries and statistics, starting from the same vector layers used in the current approach. The management of the statistical data has been completely redesigned to save the statistics in database tables, making easier to create new knowledge by mixing and experimenting with data on a wide basis. This enhancements prepare the system for further developments in the field of landscape archaeology and big data management.
\cleardoublepage
    \chapter{Case study: geostatistics on neolithic settlements in Tavoliere}

    \begin{chaptersum}
        Sommario in italiano
    \end{chaptersum}

    The geostatistical system described in this chapter has been structured using the processed spatial data collected during almost ten years of surveying of the settlements in the Tavoliere plain. The existence of buried settlements ranging from neolithic period to middle ages has been proved by field and aerial surveys and by historical records. Given the variety and widespread distribution of the settlements across about \SI{150}{\hectare}, aerial and geophysical methods assured the best results, and heavily contributed to collect the data useful to an accurate landscape archaeology study \cite[pp.~45--48]{remote-ciminale}.

    \section{General features of the settlements}
        As extensively reported in literature \cite{intro-tavoliere}, the 256 sites form one of the densest concentrations of prehistoric settlements in Europe, lying within a plain approximately \SI{50}{\kilo\meter} by \SI{80}{\kilo\meter} at its broadest and longest points. The natural boundaries of this area are marked from rivers Fortore and Ofanto --- north and south --- and the Gulf of Manfredonia and Appennine foothills --- east and west.\\
        The settlements are large or small villages, normally surrounded by one or more ditches, with the interior generally filled with a number of internal compounds. The majority of the sites occupied level ground. Refer to published materials for further details.

        \subsection{Published data and the work of J. D. B. Jones}

            \subsubsection{Spatial analysis}

        \subsection{Modern spatial investigations}
            \begin{table}
    \centering
    \begin{tabular}{cccc}
        \toprule
        Layer type & Description          & Features & Scale \\
        \otoprule
        raster     & magnetic prospection grid & --- & TODO \\
        raster     & magnetic prospection & ---      & TODO \\
        raster     & aerial photography   & ---      & TODO \\
        raster     & WMS SIT Puglia       & ---      & TODO \\
        raster     & WMS Geoportale Nazionale & ---  & TODO \\
        raster     & IGM cartography      & ---      & 1:25000 \\
        raster     & geologic cartography & ---      & TODO \\
        vector     & magnetic prospection vertexes & --- \\
        vector     & ditches / compounds / areas & 1300   & --- \\
        \bottomrule
    \end{tabular}
    \caption[List of layers in Tavoliere neolithic GIS project]{Details of the layers in the Tavoliere GIS project.
    \label{tab:layers}
    }
\end{table}

% viene prima inserito magnetogramma con griglia in unico file raster, poi derivato magnetogramma senza griglia
% georeferenziazione viene operata su magnetogramma con griglia

            After the post-processing, all the data have been placed in a GIS environment, currently containing XXX different raster layers and a single vector layer (\fref{tab:layers}). These vector layers refer to the latest study in chronological order, carried out by dott.ssa Angela Laterza in 2013,
            % TODO: add bib reference to Angelica's thesis
            who completely digitized the structures contained in 23 neolithic settlements. The whole digitalization process took as long as 320 hours and consisted in manually tracing the borders of the visible structures (\emph{ditches} and \emph{compounds}) and, as a separate geometry, the respective area for each of them; the operator has been employed for about two months, and has produced about 1300 geometries; \fref{fig:scheme-derive} shows the whole process.

            \begin{figure}[htb]
                \resizebox{1.1\textwidth}{!}{%
                    \input{tab/tab-geo-workflow}
                }
                \caption[Data deriving workflow for the Tavoliere project]{Data derivation workflow for the case study. Raster layers (\textsf{rs}) are the source of all derived vector data (\textsf{vt}). Bold lines represents a manual process, while dashed ones an automated process. Perimeter as numeric value have been calculated and saved in derived geometries.}
                \label{fig:scheme-derive}
            \end{figure}

    \section{The GIS approach to statistics}
        The first step in statistical analysis at any level is to build a suitable set of data. In this case study, vector shapes of ditches and compounds can be considered as \emph{primary} data, while area geometries as \emph{derived} data.
        The statistical and spatial approach to the study of neolithic --- or, in general, prehistoric --- settlements data is well documented in literature \cite{arch-location-model}, and the strict relation between the morphology of the natural environment and the ancient societies enforces the use of GIS systems when deriving data.\\
        Moving from the standard geophysical survey workflow to a spatial analysis context during the study of the settlements has generated some necessities, which need to be added to the critical points already defined in \fref{sec:gis-data-management}:

        \begin{description}
            \item[automate derived data creation] the set of derived data may be generated automatically, since it consists of geometries describing the same characteristic on every shape (e.g. the internal area of a ditch, for all ditches);
            \item[automate calculations] most of the processes operated on data take a feature of the element (e.g. the perimeter), save it in a database field, retrieves all of them and make calculations; this is a repetitive task which could be done more efficiently;
            \item[platform independent results] the results must be reproducible by any researcher on any platform, and the dynamics of calculations must be public; the structure of the project should enable other researchers to contribute and improve the algorithms.
        \end{description}

        Given that, some objectives have been defined to improve the current workflow and speed up the process, trying to gain in the same time more precision, accuracy and reproducibility; the following subsections will explain how every objective has been achieved in detail.

        \subsection{Development of the open source webGIS interface}

        \subsection{Bulk distinguishing ditches and compounds}
            In the current approach, ditches and compounds geometries are distinguished by attributes (the \textsf{Ditch\_comp} field in the Shapefile is respectively set to $1$ or $2$), manually added by the operator to each structure in the Shapefile.

            \begin{wrapfigure}{l}{0.5\textwidth}
                \centering
                \begin{tikzpicture}[x=1mm,y=1mm,scale=0.005]
                    \input{tab/dot-flow-map}
                \end{tikzpicture}
                \caption[Flow chart: the logic of bulk distinguishing ditches and compounds]{If any of the ditches and compounds are sharing the same color (class), class total number $k$ must be changed. At the end, geometry type is saved as an attribute in the shapefile.}
                \label{fig:flow-map}
            \end{wrapfigure}

            This operation has been semi-automated using the Jenks Natural Breaks classification method \cite{jenks1977}. Virtually any classification could have been chosen (quantile, standard deviation), but Jenks Natural Breaks has gained popularity in the last 20 years as a tool for coloring map objects based on objects properties (\emph{choropleth maps}), and it is a well known and tested tool\footnote{Some criticism has recently raised around the use of Jenks Natural Breaks for the classification of \emph{all} kinds of data; alternative solutions have been found for some distributions --- as heavy-tailed ones \cite{jenks-tail} --- but this is not the case, and Jenks has been experimentally proved as the most appropriate choice.}. It aims to present a series of break values that best represent the actual breaks observed in the data as opposed to some arbitrary classificatory scheme (i.e. equal interval); in this way the actual clustering of data values is preserved.

            In the case study, geometries have been classified by perimeter values (calculated and saved automatically from ditches and compounds shapes). The different coloring of the geometries on the map based on the newly created classes (\fref{fig:jenks-color}) helps the user to distinguish ditches from compounds at a glance.

            The process is semi-automated since the user have to manually select the ditches and, if necessary, change the class numbers to have a good fit. The flowchart in \fref{fig:flow-map} reports the logic of the process. At the end, the geometry types are saved as text attribute (literally \textsf{ditch} or \textsf{compound}) in a table containing the results of the processing (one row for each geometry); \fref{tab:jnb-results} shows sample data from the \emph{Anglisano} settlement.

            \begin{figure}[H]
                \centering
                \rotatebox{180}{
                    \begin{tikzpicture}[x=1mm,y=1mm,scale=0.2]
                        \input{img/jenks}
                    \end{tikzpicture}
                }
                \caption[A choropleth map of \emph{Anglisano} settlement using Jenks Natural Breaks]{A choropleth map of \emph{Anglisano} settlement created using Jenks Natural Breaks algorithm on perimeters automatically derived from geometries (using default value for classes number, 5). Color difference between ditches and compounds is well-rendered.}
                \label{fig:jenks-color}
            \end{figure}

            \begin{table}
                \centering
                \begin{tabular}{ccccc}
    \toprule
    ID & shapefile ID & perimeter & class & type\\
    \otoprule
    1 & 1 & 122.792532 & 3 & compound\\
    2 & 1 & 89.612125 & 2 & compound\\
    4 & 1 & 156.587759 & 3 & compound\\
    16 & 1 & 1606.964845 & 6 & ditch\\
    \bottomrule
\end{tabular}

                \caption[Sample geometry classification results from Anglisano settlements using Jenk Natural Breaks method]{Sample results from the classification of Anglisano settlement's structures by perimeter using the Jenks Natural Breaks method. The kind of structure is saved as text attribute in the \textsf{type} column. The \textsf{shapefile\_id} column binds the geometries to the respective settlement.}
                \label{tab:jnb-results}
            \end{table}

        \subsection{Derive and measure ditches and compounds areas}
            After distinguishing ditches from compounds, efforts have been focused on geometric data deriving. One of the most useful operations when analyzing settlements is to calculate the area enclosed in ditches or compounds. Unlike some cases published in literature, facing the area calculation of rectangular shaped \emph{longhouses} \cite{spatial-south-europe}, the present study has to deal with buried, irregularly shaped, rounded compounds.

            The simplest possible approach to compound's area calculation is subtracting from the whole compound area the area occupied by the compound wall, obtaining the extension of the enclosed space. This method has some drawbacks, mostly the needing to fix the limits of the doorway. A sample result using this approach can be seen in \fref{fig:comp}. Iterating over this method on all the compounds would have generated a consistent error so this method has been discarded.

            The problem has been solved applying a common analytical geometry procedure, considering the compound's internal surface as the envelope of the compound wall's points nearest to an inner point (the compound's \emph{centroid}). Compound's inner side points have been captured as the intersection between the compound's wall and a new segment binding the centroid with a point external to the compound. Rotating this external points by \SI{1}{\degree}, the segment itself is rotated and the intersection with the innermost point of the compound's wall is registered. This process is far better explained by %\fref{fig:comp-area}

            \begin{figure}[H]
                \centering
                \subfloat[Compund's wall surface]{
                    \label{fig:comp-wall}
                    \begin{tikzpicture}[x=1mm,y=1mm,scale=0.05]
                        \path[draw=black,miter limit=4.00,line width=0.8pt,name path=compound,postaction={decorate}]
    (379.0000,227.0000) --
    (376.0000,227.0000) -- (372.0000,224.0000) -- (367.0000,221.0000) --
    (364.0000,218.0000) -- (362.0000,213.0000) -- (359.0000,210.0000) --
    (388.0000,188.0000) -- (427.0000,167.0000) -- (464.0000,156.0000) --
    (481.0000,153.0000) -- (511.0000,150.0000) -- (541.0000,153.0000) --
    (570.0000,161.0000) -- (587.0000,168.0000) -- (604.0000,177.0000) --
    (609.0000,182.0000) -- (617.0000,188.0000) -- (632.0000,203.0000) --
    (644.0000,221.0000) -- (653.0000,239.0000) -- (658.0000,260.0000) --
    (660.0000,281.0000) -- (658.0000,302.0000) -- (652.0000,321.0000) --
    (645.0000,338.0000) -- (644.0000,341.0000) -- (637.0000,365.0000) --
    (627.0000,389.0000) -- (613.0000,410.0000) -- (595.0000,426.0000) --
    (573.0000,441.0000) -- (551.0000,452.0000) -- (546.0000,453.0000) --
    (524.0000,453.0000) -- (478.0000,446.0000) -- (446.0000,432.0000) --
    (439.0000,432.0000) -- (420.0000,426.0000) -- (403.0000,419.0000) --
    (388.0000,408.0000) -- (374.0000,395.0000) -- (370.0000,387.0000) --
    (367.0000,383.0000) -- (365.0000,372.0000) -- (364.0000,360.0000) --
    (365.0000,353.0000) -- (366.0000,353.0000) -- (373.0000,353.0000) --
    (380.0000,356.0000) -- (386.0000,359.0000) -- (392.0000,362.0000) --
    (398.0000,368.0000) -- (401.0000,374.0000) -- (404.0000,380.0000) --
    (407.0000,387.0000) -- (416.0000,396.0000) -- (434.0000,410.0000) --
    (454.0000,419.0000) -- (475.0000,425.0000) -- (498.0000,426.0000) --
    (502.0000,429.0000) -- (511.0000,434.0000) -- (520.0000,437.0000) --
    (529.0000,437.0000) -- (539.0000,437.0000) -- (548.0000,434.0000) --
    (557.0000,429.0000) -- (564.0000,425.0000) -- (571.0000,417.0000) --
    (577.0000,410.0000) -- (581.0000,401.0000) -- (582.0000,396.0000) --
    (587.0000,392.0000) -- (600.0000,372.0000) -- (610.0000,350.0000) --
    (617.0000,327.0000) -- (618.0000,303.0000) -- (617.0000,281.0000) --
    (614.0000,269.0000) -- (615.0000,260.0000) -- (614.0000,246.0000) --
    (610.0000,233.0000) -- (605.0000,219.0000) -- (596.0000,209.0000) --
    (587.0000,198.0000) -- (575.0000,191.0000) -- (566.0000,186.0000) --
    (548.0000,177.0000) -- (526.0000,171.0000) -- (502.0000,170.0000) --
    (479.0000,171.0000) -- (456.0000,177.0000) -- (446.0000,182.0000) --
    (379.0000,227.0000);
 
                    \end{tikzpicture}
                }
                \hspace{0.1\textwidth}
                \subfloat[Compound's internal surface]{
                    \label{fig:comp-area}
                    \begin{tikzpicture}[x=1mm,y=1mm,scale=0.05]
                        \begin{scope}[shift={(-191.875,-81.87406)}]
  \begin{scope}[draw=black,fill=c3cad6d,line join=round,line width=0.208pt]
    \path[draw,fill] (192.0000,450.0000) -- (214.0000,213.0000) --
      (213.0000,220.0000) -- (214.0000,228.0000) -- (216.0000,235.0000) --
      (219.0000,242.0000) -- (224.0000,249.0000) -- (229.0000,254.0000) --
      (232.0000,256.0000) -- (233.0000,258.0000) -- (236.0000,261.0000) --
      (238.0000,262.0000) -- (241.0000,264.0000) -- (244.0000,264.0000) --
      (247.0000,265.0000) -- (250.0000,264.0000) -- (252.0000,264.0000) --
      (255.0000,262.0000) -- (258.0000,261.0000) -- (260.0000,258.0000) --
      (261.0000,256.0000) -- (263.0000,253.0000) -- (264.0000,250.0000) --
      (264.0000,247.0000) -- (264.0000,244.0000) -- (263.0000,242.0000) --
      (261.0000,239.0000) -- (260.0000,236.0000) -- (273.0000,222.0000) --
      (289.0000,199.0000) -- (301.0000,173.0000) -- (307.0000,152.0000) --
      (323.0000,148.0000) -- (341.0000,140.0000) -- (356.0000,129.0000) --
      (370.0000,116.0000) -- (377.0000,106.0000) -- (428.0000,87.0000) --
      (490.0000,82.0000) -- (498.0000,83.0000) -- (501.0000,84.0000) --
      (564.0000,102.0000) -- (579.0000,110.0000) -- (591.0000,106.0000) --
      (645.0000,101.0000) -- (664.0000,121.0000) -- (684.0000,149.0000) --
      (698.0000,181.0000) -- (702.0000,194.0000) -- (742.0000,320.0000) --
      (750.0000,355.0000) -- (747.0000,358.0000) -- (736.0000,374.0000) --
      (727.0000,393.0000) -- (722.0000,412.0000) -- (720.0000,432.0000) --
      (721.0000,439.0000) -- (720.0000,439.0000) -- (713.0000,456.0000) --
      (702.0000,471.0000) -- (689.0000,483.0000) -- (674.0000,494.0000) --
      (658.0000,502.0000) -- (640.0000,506.0000) -- (625.0000,508.0000) --
      (604.0000,519.0000) -- (544.0000,536.0000) -- (482.0000,536.0000) --
      (479.0000,536.0000) -- (422.0000,537.0000) -- (348.0000,524.0000) --
      (304.0000,508.0000) -- (303.0000,507.0000) -- (232.0000,445.0000) --
      (202.0000,426.0000) -- (199.0000,430.0000) -- (195.0000,440.0000) --
      (192.0000,450.0000);
  \end{scope}
  \begin{scope}[draw=black,fill=c3cad6d,line join=round,line width=0.208pt]
  \end{scope}
  \begin{scope}[draw=black,fill=c3cad6d,line join=round,line width=0.208pt]
  \end{scope}
  \begin{scope}[draw=black,fill=c3cad6d,line join=round,line width=0.208pt]
  \end{scope}
  \begin{scope}[draw=black,fill=c3cad6d,line join=round,line width=0.208pt]
  \end{scope}
\end{scope}
\begin{scope}[shift={(-191.875,-81.87406)},draw=cff0000,line width=0.800pt]
\end{scope}
 
                    \end{tikzpicture}
                }
                \caption[Deriving compound area from difference between total occupied surface and wall surface]{The area derived from the difference between compound's total occupied surface and the wall surface is not suitable for analysis, since doorstep is not set adequately.}
                \label{fig:comp}
            \end{figure}

        \subsection{Automating compounds orientation discovering}
            % compounds orientation derived from external rectangle orientation

        %\subsection{Inner/outer compounds}
\cleardoublepage
    \chapter{Results: statistical data and webGIS interface}

    \vspace{0.06\textheight}
    \begin{chaptersum}
        Sommario in italiano
    \end{chaptersum}

    \section{Statistical results}
        \subsection{Ditches and compounds classification}
            The method described in \fref{sec:distinguish} has enabled the operator to write once all the attributes in every ditch and compound, distinguishing them. The procedure requires around \SI{1}{\minute} for each settlements, compared to nearly \SI{1}{\hour} in a non-automatized approach (depending on the number of geometries in the settlement). The whole set of 11 settlements containing 199 geometries has been analyzed in around 10 minutes. The plots in \fref{fig:graph-num-compound} and \fref{fig:graph-num-ditch} compare the number of ditches and compounds distinguished as such from the human brain-eye system as reported in \cite{laterza} and the calculated values, showing a perfect matching between the two datasets. The settlements are described using their site code (see \fref{tab:layers}).

            \begin{figure}[H]
                \centering
                \begin{tikzpicture}
                    \input{tab/graph-num-compound}
                \end{tikzpicture}
                \caption[The number of compounds in \cite{laterza} compared to the results of the proposed method.]{The number of compounds automatically calculated for all the analyzed settlements, compared with the results reported in \cite{laterza}.}
                \label{fig:graph-num-compound}
            \end{figure}

            \begin{figure}[H]
                \centering
                \begin{tikzpicture}
                    \pgfplotstableread{tab/raw/number-ditch}{\loadedtable}
\small

\begin{axis}[
    width=1\textwidth,
    height=0.5\textwidth,
    ybar,
    ymin=0,
    enlarge x limits,
    xlabel=settlements,
    xlabel shift=-5pt,
    ylabel=ditches $(n)$,
    xtick={1,...,11},   % indicates xticks going from 1 to 11
    xticklabels from table={\loadedtable}{set},
    xticklabel style={rotate=45,font=\scriptsize},
    xticklabel shift=-3pt,
    legend style={font=\scriptsize,draw=black!40}
]
    \addplot table[x=id, y=laterza] from \loadedtable;
    \addplot table[x=id, y=current] from \loadedtable;
    \legend{Lat13,current}
\end{axis}

                \end{tikzpicture}
                \caption[The number of ditches in \cite{laterza} compared to the results of the proposed method.]{The number of compounds automatically calculated for all the analyzed settlements, compared with the results reported in \cite{laterza}.}
                \label{fig:graph-num-ditch}
            \end{figure}

            Having all the statistics registered in a database, is very easy to extract further data; the \fref{fig:graph-perim-class}, for example, shows the frequency of perimeter's classes for all the compounds; it si evident that most of the analyzed compounds are contained in the first class of perimeters, ranging from \SI{96.6}{\meter} to \SI{103}{\meter}.

            \begin{figure}[H]
                \centering
                \begin{tikzpicture}
                    \pgfplotstableread[col sep=comma]{tab/raw/perim-class.csv}{\loadedtable}
\small

\begin{axis}[
    width=0.5\textwidth,
    height=0.5\textwidth,
    ybar,
    ymin=0,
    enlarge x limits,
    xlabel=perimeter classes,
    xlabel shift=-5pt,
    ylabel=frequency,
    xtick={1,...,5},   % indicates xticks going from 1 to 11
    xticklabels from table={\loadedtable}{class},
    xticklabel style={font=\scriptsize},
    xticklabel shift=3pt
]
    \addplot table[x=id, y=freq] from \loadedtable;
\end{axis}

                \end{tikzpicture}
                \caption[test]{test}
                \label{fig:graph-perim-class}
            \end{figure}

        \subsection{Derived areas}
            % misc stats:
            % min, max, avg area

            The method described in \fref{sec:comp-area} has produced the results shown in the rightmost column of \fref{fig:graph-area} for each analyzed settlement. The results obtained with the automated method are accurate enough to be closely compared with ones drawn by hand using GIS tools from the operator, as published in \cite{laterza}.
            As will be reported in the next section, the \emph{Salerno 13} settlement shows a small discordance with the previously published data.

            \begin{figure}[H]
                \centering
                \begin{tikzpicture}
                    \pgfplotstableread[col sep=comma]{tab/raw/aree.csv}{\loadedtable}
\small

\begin{axis}[
    width=1\textwidth,
    height=0.5\textwidth,
    ybar,
    ymin=0,
    enlarge x limits,
    xlabel=settlements,
    xlabel shift=-5pt,
    ylabel=area $(\si{\meter\squared})$,
    xtick={1,...,11},   % indicates xticks going from 1 to 11
    xticklabels from table={\loadedtable}{set},
    xticklabel style={rotate=45,font=\scriptsize},
    xticklabel shift=-3pt,
    legend style={font=\footnotesize,draw=black!40}
]
    \addplot table[x=id, y=laterza] from \loadedtable;
    \addplot table[x=id, y=current] from \loadedtable;
    \legend{Laterza 2013,current}
\end{axis}

                \end{tikzpicture}
                \caption[test]{test}
                \label{fig:graph-area}
            \end{figure}

        \subsection{Derived perimeters}
            % misc stats
            % mix, max, avg perimeter
            
            As explained in \fref{sec:comp-area}, the perimeter for each compound is directly derived from the geometry of its area. The error seen in the previous section for the \emph{Salerno 13} settlement has repercussions on the perimeter. This behavior can be attributed to an error in the area derivation using the automated method: this prevented the construction of the area geometry for 8 compounds (out of 36), modifying the total area value for the settlement (and the relative derived perimeter). This is the only known error in the proposed method.

            \begin{figure}[H]
                \centering
                \begin{tikzpicture}
                    \pgfplotstableread[col sep=comma]{tab/raw/perim.csv}{\loadedtable}
\small

\begin{axis}[
    width=1\textwidth,
    height=0.7\textwidth,
    ybar,
    ymin=0,
    enlarge x limits,
    xlabel=settlements,
    xlabel shift=-5pt,
    ylabel=perimeter $(m)$,
    xtick={1,...,11},   % indicates xticks going from 1 to 11
    xticklabels from table={\loadedtable}{set},
    xticklabel style={rotate=45,font=\scriptsize},
    xticklabel shift=-3pt,
    legend style={font=\footnotesize,draw=black!40}
]
    \addplot table[x=id, y=laterza] from \loadedtable;
    \addplot table[x=id, y=current] from \loadedtable;
    \legend{Laterza 2013,current}
\end{axis}

                \end{tikzpicture}
                \caption[test]{test}
                \label{fig:graph-perim}
            \end{figure}

        \subsection{Orientation discovering}
            % misc stats
            % all open compounds = 155
            % min, max, avg access length
            % is there any relation between orientation and access length?

            Automating the discovering of the compounds' orientations has been the hardest task, since involves the derivation of multiple geometries for each compound (see \fref{sec:orientation}). The plotted results in \fref{fig:graph-orient} show a general trend in the orientations calculated in this work, very similar to the one published in \cite{laterza}. However, the significance of this calculation can be improved, since presenting each orientation as an interval of values (e.g.\ NE contains all the values ranging from $0$ to \SI{45}{\degree}) is an operation of classification itself, and has caused a loss of detail. This has been done to register values comparable to the ones defined in \cite{laterza}. At the current status, nothing can be done to prove which measurement is nearest to reality, but the assumption that both datasets share the same trends seems a result good enough to state that the automated method is fully functional.

            \begin{figure}[H]
                \centering
                \begin{tikzpicture}
                    \pgfplotstableread[col sep=comma]{tab/raw/orient.csv}{\loadedtable}
\small

\begin{axis}[
    width=1\textwidth,
    height=0.5\textwidth,
    ybar,
    ymin=0,
    enlarge x limits,
    xlabel=orientation,
    xlabel shift=-5pt,
    ylabel=frequency,
    xtick={1,...,8},   % indicates xticks going from 1 to 11
    xticklabels from table={\loadedtable}{orient},
    xticklabel style={rotate=45,font=\scriptsize},
    xticklabel shift=-3pt,
    legend style={font=\footnotesize,legend pos=north west,draw=black!40}
]
    \addplot table[x=id, y=laterza] from \loadedtable;
    \addplot table[x=id, y=current] from \loadedtable;
    \legend{Laterza 2013,current}
\end{axis}

                \end{tikzpicture}
                \caption[test]{test}
                \label{fig:graph-orient}
            \end{figure}

        \subsection{The web interface structure\label{sec:webgis}}
            The developed web interface consists of few pages (to avoid confusion when navigating); briefly listed:

                \begin{description}
                    \item[Shapefiles list] the start page, lists all the uploaded Shapefiles and has a button redirecting to the upload page; if no Shapefile has been uploaded, the list is empty;
                    \item[upload page] enables the user to upload a new Shapefile, which is consequently saved to the software's internal database;
                    \item[Shapefile details] the most relevant page, shows a map of the selected Shapefile on the left and a list of all possible statistics on the right; every statistic has its own button, which starts the calculation when clicked; if a statistic has been calculated, the relevant row has a green background, or red on the contrary; the upper right part of the page contains buttons to manage the current Shapefile, enabling the deletion, statistics plotting or downloading.

                    \item[Jenks classification] while all other algorithms do not need any interaction with the user, the Jenks Natural Breaks classification sometimes needs the number of classes $k$ to be redefined by the operator; for this purpose a wizard has been created; it is launched pressing the button associated with the row containing the number of ditches or compounds in the Shapefile details page; the user can change the $k$ value and the colored map representing classified values is refreshed on the fly.
                \end{description}

        \subsection{Implementation specifications}
            \subsubsection{Data processing: operator-driven vs automated GIS methods}
            \subsubsection{Scientific reproducibility: the open source approach advantage}
\cleardoublepage
    \chapter{Conclusions}

    \begin{chaptersum}
        \blindtext[2]
    \end{chaptersum}

    \section{The new approach to GIS data management}
        % brief summary of what have been done
        % importance of open source software, methods, science
        % more tools -> more data -> cannot afford more researches -> better tools
        % find regularities in natural/human phaenomena
        % comprehension of complex phaenomena using big data for large-scale analysis

    \section{Further developments}
        % fix glitches in the area drawing algorithm
        % write more performant code over big amount of data
        % export data in a common spatially-enabled format (Shapefile, geoJSON)
        % import data from alternative source (geoJSON, API)
\cleardoublepage
    %******************************************
    % Materiale finale
    %******************************************
    \chapter*{Acknowledgments}
    A thesis is not the coolest place to thank people --- so, I'm sorry guys. Everyone in this page has given me something: useful tips, thoughts, insights, voluntarily or involuntarily. That said, if this work is crappy, it is partially your fault. Mixed thanks follows, without any sorting effort.
   
    I really appreciate the trust prof. Marcello Ciminale and dott.ssa Mariangela Noviello put in this work, and I am thankful for their patience and efforts in getting the best results.\\
    Although the free software projects will be subsequently mentioned, a special thank goes to Francesco Lovergine for introducing me to \LaTeX~and FOSS4G, and to Giuseppe Vallarelli for --- literally --- forcing me to install GNU/Linux in 2006, and for having left me without any working internet connection thereafter.\\
    I wish to thank anyone contributed to my archaeological experiences: Austacio Busto, Raffaella Palombella, and the O.I.A. team: Enrica, Ginevra, Patrizia.\\
    The contributions given by Domenico Giusti to this work in terms of papers research has been important, thank you Dome!\\
    I am grateful to my second family: Letizia, Vincenzo, Giuseppe, Alessia, Domenico, Chiara, Francesco, Francesco, Dario, Pasquale for their constant support and sincere friendship during these years.\\
    A special mention goes to Alessandro for his technical help, psychological support, housing facility and a lots of other things I will omit here for the sake of brevity.\\
    Thanks to Stuart Eve and the L.-P. Archaeology team for their support during the internship: just a little of this work could have been done without it.\\
    Thanks to Tina for all the abovementioned things together.

    \subsection*{Free software community}
        Simply put, very little of the current work could have been realized without standing on the shoulders of giants, as Isaac Newton said. The giants in this case are the wonderful people beyond some great \emph{free as in freedom} software. My thankfulness goes to all the guys who contributed to realize the computer programs and projects I have relied on: ArchLinux, Debian GNU/Linux, VIM, \LaTeX~and TikZ, GraphViz, GNUPlot, QuantumGIS, GDAL and GEOS, OpenJump, PostGIS, SpatiaLite, TileMill, OpenStreetMap, tmux, the K Desktop Environment, Python, (geo)Django, Twitter Bootstrap, the Archaeological Recording Kit, Nginx, LeafletJS, PHPMyAdmin, MySQL, InkScape, LibreOffice.
\cleardoublepage
    \printbibliography\cleardoublepage
    \appendix
    \addtocontents{toc}{\protect\setcounter{tocdepth}{1}}
    \addcontentsline{toc}{part}{\appendixname} 
    \chapter{Code}
\label{app:code}
\lstset{
    language=Python,
    showspaces=false,
    basicstyle=\scriptsize,
    numbers=\left,
    numberstyle=\tiny,
    stepnumber=1,
    numbersep=5pt,
    commentstyle=\color{black!60}
}

\section{Jenks Natural Breaks classification}
\label{sec:code-jenks}

    \begin{lstlisting}
    # [A] check if features for this shp already exist in helper table
    if cur_shp.feature_set.count() == HelperDitchesNumber.objects \
        .filter(shapefile_id=cur_shp_id) \
            .count():
        cur_feat = HelperDitchesNumber.objects.filter(
            shapefile_id=cur_shp_id)
    else:
        # create helping features and fill in perimeter for each
        cur_shp_geom = get_geos_geometry(cur_shp)
        cur_shp_geom.set_srid(4326)
        cur_shp_geom.transform(3857)
        for feature in cur_shp_geom:
            if feature.geom_type == 'Polygon':
                feature = MultiPolygon(feature)

            # get perimeter in Shapefile's projection
            proj_feature = feature
            proj_feature.set_srid(cur_shp_geom.srid)
            proj_feature.transform(cur_shp.proj)

            new_feat = HelperDitchesNumber(
                poly=feature,
                shapefile_id=cur_shp_id,
                perimeter=proj_feature.length)
            new_feat.save()

        cur_shp.stat_ditch_comp = cur_shp.stat_ditch_comp = True
        cur_shp.save()

        cur_feat = HelperDitchesNumber.objects.filter(
            shapefile_id=cur_shp_id)

    # [B] check if this shapefile has custom class number defined
    if cur_shp.classes:
        class_num = cur_shp.classes
    else:
        class_num = settings.GEOSTAT_SETTINGS[
            'jenk_natural_breaks_classes']

    # [C] get all perimeters as list
    perimeters = cur_feat.values_list('perimeter', flat=True)
    perim_list = []
    for x in perimeters:
        perim_list.append(x)

    # [D] calculate Jenks Natural Breaks, save in shapefile and features
    jnb_classes_list = get_jenks_breaks(perim_list, int(class_num))
    Shapefile.objects.filter(id=cur_shp_id) \
        .update(jnb=jnb_classes_list)

    # [E] fill in the class for each feature of the helper layer
    for feature in cur_feat:
        class_val = classify(feature.perimeter, jnb_classes_list)
        feature.class_n = class_val
        feature.save()
    \end{lstlisting}

\section{Get round vertexes external to compound geometry}
\label{sec:code-vertexes}

    \begin{lstlisting}
        def get_round_vertex(angle, radius,
            point_x, point_y, projection=3857, rotation=0):
            # Returns a list of GEOS points representing a circle around the point
            # (point_x, point_y) of radius `radius`, separated of an `angle` value (in
            # degrees) from the next point.
            vertex_list = []
            angle_orig = angle
            while angle <= 360:
                vertex_x = point_x + radius * cos(radians(angle))
                vertex_y = point_y + radius * sin(radians(angle))
                if rotation > 0:
                    x = ((vertex_x - point_x)*cos(rotation) - (vertex_y - point_y)
                         * sin(rotation)) + point_x
                    y = ((vertex_x - point_x)*sin(rotation) + (vertex_y - point_y)
                         * cos(rotation)) + point_y
                    vertex_list.append(Point(x, y, srid=projection))
                else:
                    vertex_list.append(Point(vertex_x, vertex_y, srid=projection))
                angle += angle_orig

            # test printing
            testpoints = MultiPoint(vertex_list)
            print(testpoints.geojson)

            return vertex_list
    \end{lstlisting}

\section{Get innermost intersection of segment and compound}
\label{sec:code-intersection}

    \begin{lstlisting}
    def check_nearest_point(coords_tuple, srid, refer_pt, defined_nearest):
        # Checks if the point generated by a tuple of coordinates (x, y) of reference
        # system ID `srid` is near to a reference point `refer_pt` more of an already
        # defined value `defined_nearest`; if true, updates the `refer_pt` list with
        # the nearest GEOS Point `refer_pt[0]` and the distance `refer_pt[1]`.
        cur_point = Point(coords_tuple, srid=srid)
        if cur_point.distance(refer_pt) < defined_nearest[1]:
            defined_nearest = [cur_point, cur_point.distance(refer_pt)]
        else:
            pass
        return defined_nearest
    \end{lstlisting}

\section{Deriving areas}
\label{sec:code-area}

    \begin{lstlisting}
    # [A] get convex hull and its centroid
    feat_convex_hull = feature.poly.convex_hull
    feat_centroid = feat_convex_hull.centroid

    # [B] feature's hull farthest point from centroid
    max_point = Point(
        feat_convex_hull.extent[2],
        feat_convex_hull.extent[3],
        srid=3857)

    radius = max_point.distance(feat_centroid)

    # get vertexes in a circle (center=centroid), every n angle
    vertexes_list = get_round_vertex(1, radius,
                                     feat_centroid.x,
                                     feat_centroid.y,
                                     3857)

    # for each point in vertex list
    for point in vertexes_list:

    # create new line between point and centroid
        line = LineString(feat_centroid, point, srid=3857)
        # line intersects geometry: get point nearest to centroid
        try:
            intersection_line = line.intersection(feature.poly)
        except GEOSException:
            pass
        if intersection_line.num_coords == 0:  # no intersection
            pass
        # intersection in 1 point
        elif intersection_line.num_coords == 1:
            area_points_list.append(Point(
                intersection_line.coords[0]))
        # intersection in 2 or more points
        elif intersection_line.num_coords >= 2:
            nearest_point = [None, 10000000]
            # intersection generates a MultiLineString (> 2 pts)
            if intersection_line.geom_type == 'MultiLineString':
                for multiline_tuple in intersection_line.tuple:
                    for coords_tuple in multiline_tuple:
                        nearest_point = check_nearest_point(
                            coords_tuple, 3857,
                            feat_centroid,
                            nearest_point)
                area_points_list.append(nearest_point[0].tuple)
            # intersection generates a LineString (2 pts)
            else:
                for coords_tuple in intersection_line.tuple:
                    nearest_point = check_nearest_point(
                        coords_tuple, 3857,
                        feat_centroid,
                        nearest_point)
                area_points_list.append(nearest_point[0].tuple)

    # close polygon, get projected area and save
    area_points_list.append(area_points_list[0])
    internal_area_polygon = Polygon(area_points_list, srid=3857)

    proj_area_polygon = internal_area_polygon
    proj_area_polygon.transform(cur_shp.proj)

    if feature.type == 'compound':
        # recognize open/closed compound
        tr = settings.GEOSTAT_SETTINGS['open_compound_treshold']
        closed_limit = 360 - ((tr * 360) / 100)
        if area_points_list.__len__() > closed_limit:
            structure_open = False
        else:
            structure_open = True
    else:
        structure_open = None
    \end{lstlisting}

\section{Find compound access and orientation}
\label{sec:code-access}

    \begin{lstlisting}
    # iterate on all open compounds
    for compound in cur_shp.helpercompoundsarea_set \
            .filter(type='compound', open=True):
        # get sides and relative lengths as dictionary
        sides = get_side_dict(compound, 3857)
        # get longest side in area polygon as a LineString
        access_linestr = max(sides, key=sides.get)

        # get access lenght as projected value
        proj_access_linestr = access_linestr
        proj_access_linestr.transform(cur_shp.proj)

        # get the centroid of the access side
        feature_centroid = compound.poly.centroid

        # get compound's farthest point from centroid
        max_point = Point(compound.poly.convex_hull.extent[2],
                          compound.poly.convex_hull.extent[3],
                          srid=3857)
        radius = max_point.distance(feature_centroid)

        # draw cardinal points around the compound every 45 degree,
        # and rotate them by 12 degree to align perpendicularly to N
        cardinal_pts = get_round_vertex(
            45,
            radius,
            feature_centroid.x,
            feature_centroid.y,
            3857,
            12)

        # create "cake slices" using cardinal points
        polygon_list = []
        for i, item in enumerate(cardinal_pts):
            points = (feature_centroid.coords,
                      item.coords,
                      cardinal_pts[i - 1].coords,
                      feature_centroid.coords)
            polygon_list.append(Polygon(points, srid=3857))
        sectors = MultiPolygon(polygon_list, srid=3857)

        # get access side centroid
        access_centroid = access_linestr.centroid
        access_centroid.transform(3857)

        # find sector containing the access centroid; get direction
        for sector in sectors:
            if sector.contains(access_centroid):
                direction = sectors.index(sector)
            else:
                pass

        # save the access LineString in a separate table
        new_compound_access = HelperCompoundsAccess(
            shapefile_id=cur_shp_id,
            comp=compound,
            poly=access_linestr,
            length=proj_access_linestr.length,
            orientation=direction
        )
        new_compound_access.save()
    \end{lstlisting}

\end{document}
